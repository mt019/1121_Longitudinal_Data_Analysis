\documentclass[UTF8,a4paper,10pt]{article}

% \documentclass[UTF8,a4paper,14pt]{article}
% \usepackage[utf8]{inputenc}
\usepackage{amsmath}
% \usepackage{algorithm,algorithmic}
\usepackage[linesnumbered,ruled,vlined]{algorithm2e}

% \usepackage{algorithmicx}
% \usepackage{algpseudocode}
\usepackage{hyperref}

% \usepackage{algpseudocode}
\usepackage{amssymb}
\usepackage{amsfonts}
%for 字體
%https://tug.org/FontCatalogue/
% \usepackage[T1]{fontenc}
% \usepackage{tgbonum}
% \usepackage[bitstream-charter]{mathdesign}
% \usepackage[T1]{fontenc}
% \usepackage{bm}#粗體
%\usepackage{boondox-calo}
\usepackage{textcomp}
\usepackage{fancyhdr}%导入fancyhdrf包
\usepackage{ctex}%导入ctex包
\usepackage{enumitem} %for在Latex使用條列式清單
\usepackage{varwidth}
\usepackage{soul} %for \ul
\usepackage{comment}%\begin{comment}\end{comment}
\usepackage{cancel}%\cancel{}
%\usepackage{unicode-math}

\usepackage[dvipsnames, svgnames, x_11names]{xcolor}

\usepackage[low-sup]{subdepth}
\usepackage{subdepth}

\newcommand{\indep}{\Perp \!\!\! \Perp}

\usepackage{amsthm}
\DeclareMathOperator{\E}{\mathbb{E}}
\DeclareMathOperator{\Var}{\textbf{Var}}
\DeclareMathOperator{\Cov}{\textbf{Cov}}
\DeclareMathOperator{\Cor}{\textbf{Cor}}
\DeclareMathOperator{\X}{\mathbf{X}}
\DeclareMathOperator{\Pro}{\mathbf{P}}
\DeclareMathOperator{\M}{\mathbf{M}}
\DeclareMathOperator{\Id}{\mathbf{I}}
\DeclareMathOperator{\Y}{\mathbf{Y}}
\DeclareMathOperator{\MSFE}{\mathbf{MSFE}}
\DeclareMathOperator{\e}{\mathbb{e}}
\DeclareMathOperator{\V}{\mathbf{V}} 
\DeclareMathOperator{\tr}{\text{tr}}
\DeclareMathOperator{\A}{\textbf{A}}
\DeclareMathOperator{\diag}{diag}
\DeclareRobustCommand{\rchi}{{\mathpalette\irchi\relax}}
\newcommand{\irchi}[2]{\raisebox{\depth}{$#1\chi$}} % inner command, used by \rchi
\DeclareMathOperator*{\argmax}{arg\,max}
\DeclareMathOperator*{\argmin}{arg\,min}

\DeclareMathSizes{20}{10}{10}{5}

\usepackage[a4paper, margin=1in]{geometry}
% \setlength\parskip{5ex}% it would be better define distance in ex (5ex) 
                         %  or in pt, pc, mm, etc (see edit below)

\setlength{\parindent}{0pt}
\usepackage{array, makecell} %


%中英文設定
%\usepackage{fontspec}
% \setmainfont{TeX Gyre Termes}
% \usepackage{xeCJK} %引用中文字的指令集
% %\setCJKmainfont{PMingLiU}
% \setCJKmainfont{DFKai-SB}





% \setmainfont{Times New Roman}
% \setCJKmonofont{DFKai-SB}
\pagenumbering{arabic}%设置页码格式
\pagestyle{fancy}
\fancyhead{} % 初始化页眉
\usepackage{advdate}

% \newcommand{\yesterday}{{\AdvanceDate[-1]\today}}

\fancyhead[C]{Longitudinal Data Analysis\quad HW 01\quad  R10A21126\quad  WANG YIFAN\quad   \today}
%\fancyhead[LE]{\textsl{\rightmark}}
%\fancyfoot{} % 初始化页脚
%\fancyfoot[LO]{奇数页左页脚}
%\fancyfoot[LE]{偶数页左页脚}
%\fancyfoot[RO]{奇数页右页脚}
%\fancyfoot[RE]{偶数页右页脚}

% \title{{Econometrics HW 05}}
% \author{R10A21126}
% \date{\today}

%\fancyhf{}
\usepackage{lastpage}
\cfoot{Page \thepage \hspace{1pt} of\, \pageref{LastPage}}

\renewcommand{\headrulewidth}{0.1pt}%分隔线宽度4磅
%\renewcommand{\footrulewidth}{4pt}

\allowdisplaybreaks
\usepackage[english]{babel}
%\usepackage{amsthm}
\newtheorem{theorem}{Theorem}[section]
\newtheorem{corollary}{Corollary}[theorem]
\newtheorem{lemma}[theorem]{Lemma}


\usepackage[most]{tcolorbox}

\definecolor{babyblue}{rgb}{0.54, 0.81, 0.94}

\newtcolorbox[auto counter]{mybox}[1]{
  % Define a new tcolorbox style with custom paragraph spacing
  before upper={\parskip=10pt},
    after upper={\parskip=10pt},
    enhanced,
    arc= 1 mm,boxrule=1.5pt,
    colframe=babyblue!80!pink,
    colback=white,
    coltitle=black,
    % colback=blue!5!white,
    attach boxed title to top left=
    {xshift=1.5em,yshift=-\tcboxedtitleheight/2},
    boxed title style={size=small,
    % frame hidden,
    colback=White},
    top=0.15in,
    % fonttitle=\bfseries,
    title= {#1},
    breakable
  }

\newtcolorbox[auto counter]{Problem}[2][]{
    enhanced,drop shadow={Pink!50!white},
    colframe=pink!80!white,
    fonttitle=\bfseries,
    title=Problem ~\thetcbcounter. #2,
    %separator sign={.},
    coltitle=black,
    colback=pink!15,
    top=0.15in,
    breakable
  }

\newenvironment{solution}
  {\renewcommand\qedsymbol{$\blacksquare$}\begin{proof}[Solution]}
  {\end{proof}}

\theoremstyle{definition}
\newtheorem{definition}{Definition}[section]

%\theoremstyle{notation}
\newtheorem*{notation}{\underline{Notation}}
%\newtheorem*{convention}{\underline{Convention}}
\newtheorem*{convention}{\underline{Convention}}

\theoremstyle{remark}
\newtheorem*{remark}{Remark}

\newenvironment{amatrix}[2]{%% [2] for 2 parameters 
  \left[\begin{array}
    %{cc\,|\,cc}
    %  {@{}*{#2}{c}\,|\,c*{#1}{c}}
     {{}*{#1}{c}\,|\,c*{#2}{c}}
}{%
  \end{array}\right]
}
% For augmented matrix  
%https://tex.stackexchange.com/questions/2233/whats-the-best-way-make-an-augmented-coefficient-matrix


% defines the paragraph spacing
\setlength{\parskip}{0.5em}


\usepackage[sorting=none, citestyle=verbose-inote,backref=true,ibidtracker=context,mincrossrefs=99,backend=biber, 
url = false,
doi = false, isbn=false,]{biblatex}

\addbibresource{R10A21126.bib}

\usepackage{graphicx}
\graphicspath{ {images/} }
\usepackage{caption}

% global change
\SetKwInput{KwData}{Input}
\SetKwInput{KwResult}{Output}
% https://tex.stackexchange.com/questions/299771/how-do-i-rename-data-from-kwdata-and-result-from-kwresult-in-begi

\hypersetup{hidelinks}


\begin{document}

Cluster Data.
\[
\{(X_{i1}, Y_{i1}), \ldots, (X_{imi}, Y_{imi});\ i = 1, \ldots, n\}
\]
where \(Y_{ij}\) and \(X_{ij} = (X_{ij1}, \ldots, X_{ijp})\) are the response variable and the \(p \times 1\) coordinate vector of the \(j\)-th individual within the \(i\)-th cluster, and \(m_i\) denotes the random cluster size, \(i = 1, \ldots, n;\ j = 1, \ldots, m_i\).

\(m_i\) and \(X_i\) are related. (informative) A parametric model is considered for each individual \(Y_{ij}\),
\[
E(Y_{ij} | X_{ij}) = u_c(X_{ij}; \theta)
\]
where \(\theta = (\theta_1, \ldots, \theta_p)^T\).

\section*{Model Assumption}

Under the validity of a fitting model, an independent cluster size

\[
E[Y_{ij} | X_{ij}, m_i] = E[Y_{ij} | X_{ij}], \quad \forall i,j
\]

is usually assumed in estimation procedures, e.g., generalized estimating equation (GEE).

In this lecture, we consider the condition that the cluster size is informative, i.e.,

\[
E[E[h_j | X_{ij}, m_i]] \neq E[E[Y_j | X_{ij}]]
\]

---

\section*{Remark}

It can be verified that an estimator derived from the GEE might produce an inconsistent estimator of \(\theta\) under the assumption of informative cluster size.

---


Let 
\[Y_i = \begin{pmatrix}
    Y_{i1}\\
    \vdots\\
    Y_{i\,m_i}
\end{pmatrix},\quad \mu_i = \begin{pmatrix}
    \mu_{i1}\\
    \vdots\\
    \mu_{i\,m_i}
\end{pmatrix},\quad V_{i} = (V_{i\,j_1 j_2});\quad \quad i = 1, \ldots, n; \quad j = 1, \ldots, m_i\]

with
\begin{align*}
    V_{i\,j_1 j_2} 
    &= \text{Cor}(Y_{ij_2}, Y_{ij_2} | X_{ij_1}, X_{ij_2})\\
    &\triangleq h(x_{ij_1},x_{ij_2}; \theta)
\end{align*}

---

In GEE, an estimator is obtained by solving the estimating equations
\[
S(\theta) = 0,
\]
where \[S(\theta) = \sum_{i=1}^{n} \left(\dfrac{\partial \mu_i}{\partial \theta}\right)V_{i}^{-1}(Y_{i} - \mu_i).\]


Because of the possible dependence between \(Y_{ij}\) and \(m_i\), the conditional expectation
\[
E\left[\left(\dfrac{\partial \mu_i}{\partial \theta}\right)V_{i}^{-1}(Y_{ij} - \mu_i)\mid (X_{i1},\ldots,X_{im_i})\right]
\]
might not equal zero, and hence, the derived estimator is not consistent.

---

\section*{Within Cluster Resampling (WCR) Approach}

(Hoffman, Sen, and Weinberg (2001))

\subsection*{Estimation Procedure}

\textbf{Step 1.} The \(q\text{th}\) subsample
\(\{(X_{1q}, Y_{1q}), \ldots, (X_{nq}, Y_{nq})\}\),
\(q = 1, \ldots, Q_1 = \prod_{i=1}^{n}m_i\), is drawn with a randomly selected individual \((X_{iq}, Y_{iq})\) from the \(i\text{th}\) cluster, \(i = 1, \ldots, n\).

\textbf{Step 2:} Based on the \(q\text{th}\) subsample in Step 1, the \(q\text{th}\) estimator, say \(\hat{\theta}_{q}\), of \(\theta\), is obtained from solving the equation
\[
S_1(\theta) = \sum_{i=1}^{n} \left(\dfrac{\partial \mu_i}{\partial \theta}\right)\hat{V}_{i\,qq}^{-1}(Y_{iq} - \mu_{iq}).
\]
where \(\hat{V}_{i\,qq}\) is a consistent estimator of \(V_{i\,qq}\).

\textbf{Step 3:} Repeat Steps 1-2. The final estimator \(\widebar{\theta}_{wcr}\) takes the average of the \(Q_i\) estimators,
\[
\widebar{\theta}_{wcr} = \frac{1}{Q_1} \sum_{q=1}^{Q_1} \widehat{\theta}_i.
\]

---
\section*{Remark:}
\begin{enumerate}
    \item The number of possible sub-samples $Q_1$ is extremely large; a reasonable number of re-sampling is often implemented in the WCR approach.
    \item Two merits of the WCR approach:
    \begin{enumerate}
        \item The within-cluster correlation does not need to be specified in the estimating equations.
        \item The estimator $\widebar{\theta}_{wcr}$ is a consistent estimator of $\theta$.
    \end{enumerate}
    \item  Drawbacks: The WCR approach is found to be computationally intensive in implementation.
\end{enumerate}

---

 Cluster-Weighted generalized estimating equation (CWGEE) approach. 

 (Williamson, J,M,...(2003))
The estimator, say $\widetilde{\theta}_{c w}$ is obtain by solving the following estimating equations:
$$
S_2(\theta) \triangleq \sum_{i=1}^n \frac{1}{m_i} \sum_{j=1}^{m_i} \frac{\partial \mu_{i j}}{\theta} \hat{V}_{i j j}^{-1}\left(Y_{i j}-\mu_{i j}\right)=0
$$

---

Remark:
$$
\begin{aligned}
\mathbb{E}\left[S_2(\theta)\right] & =\sum_{i=1}^n E\left[\frac{1}{m_i} \sum_{j=1}^{m_i} E\left[\frac{\partial \mu_{i j}}{\partial \theta} \hat{V}_{i j j}^{-1}\left(Y_{i j}-\mu_{i j}\right) \mid m_i\right]\right] \\
& =\sum_{i=1}^n E\left[\frac{\partial \mu_{i j}}{\partial \theta} \hat{V}_{i j j}^{-1}\left(Y_{i j}-\mu_{i j}\right)\right] \\
& =\sum_{i=1}^n E[\frac{\partial \mu_{i j}}{\partial \theta} \hat{V}_{i j j}^{-1} \underbrace{E\left[\left(Y_{i j}-\mu_{i j}\right) \mid X_{i j}\right]}_0]=0
\end{aligned}
$$

---

Asymptotic Equivalent between the WCR and CWGEE


Under some regularity conditions, one can derive that
$$
\hat{\theta}_1=\theta+\frac{1}{n} H^{-1}(\theta) S_{1 q}^*(\theta)+o_p\left(\frac{1}{\sqrt{n}}\right) \text {. }
$$
where 
$$S_{1 q}^*(\theta)=\sum_{i=1}^n \frac{\partial \mu_{i q}}{\partial \theta} V_{i q q}\left(Y_{i q}-\mu_{i q}\right)$$ and 
$$H_1(\theta)=E\left[\frac{\partial \mu_{i q}}{\partial \theta} V_{i j j}^{-1}\left(\frac{\partial \mu_{i q}}{\partial \theta}\right)^{\top}\right].$$

As a result,
$$
\bar{\theta}_{wcr}=\theta+\frac{1}{n} H^{-1}(\theta) \frac{1}{Q_1} \sum_{q=1}^{Q_1} S_{1 q}^*(\theta)+o_p\left(\frac{1}{\sqrt{n}}\right)
$$

Similarly
$$
\widetilde{\theta} c w=\theta+\frac{1}{n} H^{-1}(\theta) S_2^*(\theta)+o_p(1)
$$

---

Theorem. 1

Under some regularity conditions
$$
\bar{\theta}_{w c t}=\widetilde{\theta}_{c w}+o_p\left(\frac{1}{\sqrt{n}}\right).
$$

Proof: 

$$
\begin{aligned}
    \frac{1}{Q_1} \sum_{q=1}^{Q_1} S_{1q}^*(\theta)& \quad=\sum_{i=1}^n \frac{1}{Q} \cdot \sum_{q=1}^{Q_1}\left(\frac{\partial \mu_{i q}}{\partial \theta}\right) V_{i q q}^{-1}\left(Y_{i q}-\mu_{i q}\right) \\
& =\sum_{i=1}^n \frac{1}{Q}\left(\prod_{l \neq i}^n m_l\right)\left(\sum_{j=1}^{m_i}\left(\frac{\partial \mu_{i j}}{\partial \theta}\right) V_{i j j}^{-1}\left(Y_{i j}-\mu_{i j}\right)\right) \\
& =\sum_{i=1}^n \frac{1}{m_i} \sum_j^{m_i}\left(\frac{\partial \mu_{i j}}{\partial \theta}\right) V_{i j j}^{-1}\left(Y_{i j}-\mu_{i j}\right) \\
& =S_2^*(\theta)
\end{aligned}
$$
\# Efficient Estimation Method Let $m=\min \left\{m_1, \ldots, m_n\right\} \geq z$
Modified Within Cluster Resampling (MWCR) approach.

Step 1: The $q^{\text {th }}$ subsample
$$
\begin{aligned}
& \left\{\left(X_i q(m), Y_i q(m)\right), \ldots,\left(X_{n q(m)}, Y_{n q(m)}\right)\right\} . \\
& q=1, \ldots, Q_m=\prod_{i=1}^n C^{m_1}
\end{aligned}
$$
is drawn with the $q^{\text {th }}$ observation
$$
\begin{aligned}
& X_{i g}(m)=\left(X_{i q}, \ldots, X_{\text {imp }}\right)^{\top}, \\
& Y_{i q}(m)=\left(Y_{i q}, \ldots, Y_{\text {imp }}\right)^{\top}, \\
& \text { of } X_i \text { and } Y_i .
\end{aligned}
$$

Step 2: Based on the $q^{\text {th }}$ subsample in Step 1, the \(q\)-th estimator, Say $\hat{Q}_{q(m)}$ of $\theta$ is defined as the solution of
$$
S_{1 q}^{(m)}(\theta)=\sum_{i=1}^n \frac{\partial \mu_{i q}(m)}{\partial \theta} \hat{V}_{i q(m)}^{-1}\left(Y_{i q(m)}-\mu_{i q(m)}\right)=0
$$
where $\hat{V}_{i q(m)}=\left(h\left(X_{i j q}, X_{i j 2 q} ; \alpha\right)\right)$



\end{document}

