\documentclass[UTF8,a4paper,10pt]{article}

% \documentclass[UTF8,a4paper,14pt]{article}
% \usepackage[utf8]{inputenc}
\usepackage{amsmath}
% \usepackage{algorithm,algorithmic}
\usepackage[linesnumbered,ruled,vlined]{algorithm2e}

% \usepackage{algorithmicx}
% \usepackage{algpseudocode}
\usepackage{hyperref}

% \usepackage{algpseudocode}
\usepackage{amssymb}
\usepackage{amsfonts}
%for 字體
%https://tug.org/FontCatalogue/
% \usepackage[T1]{fontenc}
% \usepackage{tgbonum}
% \usepackage[bitstream-charter]{mathdesign}
% \usepackage[T1]{fontenc}
% \usepackage{bm}#粗體
%\usepackage{boondox-calo}
\usepackage{textcomp}
\usepackage{fancyhdr}%导入fancyhdrf包
\usepackage{ctex}%导入ctex包
\usepackage{enumitem} %for在Latex使用條列式清單
\usepackage{varwidth}
\usepackage{soul} %for \ul
\usepackage{comment}%\begin{comment}\end{comment}
\usepackage{cancel}%\cancel{}
%\usepackage{unicode-math}

\usepackage[dvipsnames, svgnames, x_11names]{xcolor}

\usepackage[low-sup]{subdepth}
\usepackage{subdepth}

\newcommand{\indep}{\Perp \!\!\! \Perp}

\usepackage{amsthm}
\DeclareMathOperator{\E}{\mathbb{E}}
\DeclareMathOperator{\Var}{\textbf{Var}}
\DeclareMathOperator{\Cov}{\textbf{Cov}}
\DeclareMathOperator{\Cor}{\textbf{Cor}}
\DeclareMathOperator{\X}{\mathbf{X}}
\DeclareMathOperator{\Pro}{\mathbf{P}}
\DeclareMathOperator{\M}{\mathbf{M}}
\DeclareMathOperator{\Id}{\mathbf{I}}
\DeclareMathOperator{\Y}{\mathbf{Y}}
\DeclareMathOperator{\MSFE}{\mathbf{MSFE}}
\DeclareMathOperator{\e}{\mathbb{e}}
\DeclareMathOperator{\V}{\mathbf{V}} 
\DeclareMathOperator{\tr}{\text{tr}}
\DeclareMathOperator{\A}{\textbf{A}}
\DeclareMathOperator{\diag}{diag}
\DeclareRobustCommand{\rchi}{{\mathpalette\irchi\relax}}
\newcommand{\irchi}[2]{\raisebox{\depth}{$#1\chi$}} % inner command, used by \rchi
\DeclareMathOperator*{\argmax}{arg\,max}
\DeclareMathOperator*{\argmin}{arg\,min}

\DeclareMathSizes{20}{10}{10}{5}

\usepackage[a4paper, margin=1in]{geometry}
% \setlength\parskip{5ex}% it would be better define distance in ex (5ex) 
                         %  or in pt, pc, mm, etc (see edit below)

\setlength{\parindent}{0pt}
\usepackage{array, makecell} %


%中英文設定
%\usepackage{fontspec}
% \setmainfont{TeX Gyre Termes}
% \usepackage{xeCJK} %引用中文字的指令集
% %\setCJKmainfont{PMingLiU}
% \setCJKmainfont{DFKai-SB}





% \setmainfont{Times New Roman}
% \setCJKmonofont{DFKai-SB}
\pagenumbering{arabic}%设置页码格式
\pagestyle{fancy}
\fancyhead{} % 初始化页眉
\usepackage{advdate}

% \newcommand{\yesterday}{{\AdvanceDate[-1]\today}}

\fancyhead[C]{Longitudinal Data Analysis\quad HW 01\quad  R10A21126\quad  WANG YIFAN\quad   \today}
%\fancyhead[LE]{\textsl{\rightmark}}
%\fancyfoot{} % 初始化页脚
%\fancyfoot[LO]{奇数页左页脚}
%\fancyfoot[LE]{偶数页左页脚}
%\fancyfoot[RO]{奇数页右页脚}
%\fancyfoot[RE]{偶数页右页脚}

% \title{{Econometrics HW 05}}
% \author{R10A21126}
% \date{\today}

%\fancyhf{}
\usepackage{lastpage}
\cfoot{Page \thepage \hspace{1pt} of\, \pageref{LastPage}}

\renewcommand{\headrulewidth}{0.1pt}%分隔线宽度4磅
%\renewcommand{\footrulewidth}{4pt}

\allowdisplaybreaks
\usepackage[english]{babel}
%\usepackage{amsthm}
\newtheorem{theorem}{Theorem}[section]
\newtheorem{corollary}{Corollary}[theorem]
\newtheorem{lemma}[theorem]{Lemma}


\usepackage[most]{tcolorbox}

\definecolor{babyblue}{rgb}{0.54, 0.81, 0.94}

\newtcolorbox[auto counter]{mybox}[1]{
  % Define a new tcolorbox style with custom paragraph spacing
  before upper={\parskip=10pt},
    after upper={\parskip=10pt},
    enhanced,
    arc= 1 mm,boxrule=1.5pt,
    colframe=babyblue!80!pink,
    colback=white,
    coltitle=black,
    % colback=blue!5!white,
    attach boxed title to top left=
    {xshift=1.5em,yshift=-\tcboxedtitleheight/2},
    boxed title style={size=small,
    % frame hidden,
    colback=White},
    top=0.15in,
    % fonttitle=\bfseries,
    title= {#1},
    breakable
  }

\newtcolorbox[auto counter]{Problem}[2][]{
    enhanced,drop shadow={Pink!50!white},
    colframe=pink!80!white,
    fonttitle=\bfseries,
    title=Problem ~\thetcbcounter. #2,
    %separator sign={.},
    coltitle=black,
    colback=pink!15,
    top=0.15in,
    breakable
  }

\newenvironment{solution}
  {\renewcommand\qedsymbol{$\blacksquare$}\begin{proof}[Solution]}
  {\end{proof}}

\theoremstyle{definition}
\newtheorem{definition}{Definition}[section]

%\theoremstyle{notation}
\newtheorem*{notation}{\underline{Notation}}
%\newtheorem*{convention}{\underline{Convention}}
\newtheorem*{convention}{\underline{Convention}}

\theoremstyle{remark}
\newtheorem*{remark}{Remark}

\newenvironment{amatrix}[2]{%% [2] for 2 parameters 
  \left[\begin{array}
    %{cc\,|\,cc}
    %  {@{}*{#2}{c}\,|\,c*{#1}{c}}
     {{}*{#1}{c}\,|\,c*{#2}{c}}
}{%
  \end{array}\right]
}
% For augmented matrix  
%https://tex.stackexchange.com/questions/2233/whats-the-best-way-make-an-augmented-coefficient-matrix


% defines the paragraph spacing
\setlength{\parskip}{0.5em}


\usepackage[sorting=none, citestyle=verbose-inote,backref=true,ibidtracker=context,mincrossrefs=99,backend=biber, 
url = false,
doi = false, isbn=false,]{biblatex}

\addbibresource{R10A21126.bib}

\usepackage{graphicx}
\graphicspath{ {images/} }
\usepackage{caption}

% global change
\SetKwInput{KwData}{Input}
\SetKwInput{KwResult}{Output}
% https://tex.stackexchange.com/questions/299771/how-do-i-rename-data-from-kwdata-and-result-from-kwresult-in-begi

\hypersetup{hidelinks}

% \begin{equation*}
%   \begin{aligned}
%   \end{aligned}
% \end{equation*}

% \begin{mybox}{}
% \end{mybox}


% \begin{Problem}[]{}
% \end{Problem}

% \begin{solution}\,
% \end{solution}
  

% \begin{enumerate}[label=(\alph*)]
% \end{enumerate} 

% \setcounter{section}{3} 
% \setcounter{theorem}{3}

% \begin{theorem}\label{thm:3.4}
%   If $E = \bigcup_{k} E_k$ is a countable union of sets, then $|E|_e \leq \sum_{k} |E_k|_e$.
%   \end{theorem}

%  \footcite[][42]{Wheeden_Zygmund_2015}

\begin{document}


% \begin{mybox}{}


% \end{mybox}



% Q1
\section*{Transition Models}

$t_{ij}^{\prime} s$ are assumed to be equally spaced.


Let $H_i=\left\{y_k, k=1, \cdots, j-1\right\}$. 

Consider \[f(y_{ij}\mid H_{ij},\alpha,\beta) = \exp\left\{\frac{y_{ij}-\psi(\theta_{ij})}{\phi}+c(y_{ij},\phi)\right\},\] where \(\psi(\theta_{ij})\) and \(c(y_{ij},\phi)\) are known functions. 

One has 
$$\mu_{ij}^c=E\left[y_{ij} \mid H_{ij}\right]=\psi^{\prime}\left(\theta_{ij}\right)$$ 
and 
$$V_{ij}^c=V\left[y_{ij} \mid H_{ij}\right]=\psi''\left(\theta_{ij}\right) \phi$$ 

with

\[h(\mu_{ij}^c) = x_{ij}^T\beta +\sum_{r=1}^{s}f_r(H_{ij};\alpha)\text{ for suitable functions }f_r(\cdot)'s,\]
and

\[v_{ij}^c = v(\mu_{ij}^c)\phi.\]


\begin{Problem}[]{Fitting transition models: (A markov model of order $q$ )}


 

By
\[L_i(y_{i1},\cdots,y_{im_i}) = f(y_{i1},\cdots,y_{iq})\prod_{j=q+1}^{m_i}f(y_{ij}\mid y_{i\, j-1},\cdots,y_{i \, j-q}), i = 1,\cdots,n,\]

one can get the likelihood function 

\[L(\alpha,\beta) = \prod_{i=1}^{n}f(y_{i1},\cdots,y_{iq})\prod_{j=q+1}^{m_i}f(y_{ij}\mid H_{ij},\alpha,\beta),\]
where 

\[H_{ij} = \{y_{i \, j-1},\cdots,y_{i \, j-q}\}.\]

Since the term $f\left(y_{i1}, \cdots, y_{iq}\right)$ is always unavailable, the estimators of $(\alpha, \beta)$ are obtained via maximizing the conditional likelihood 

$$\prod_{i=1}^{n} \prod_{j=q+1}^{m_i} f\left(y_{ij}\mid H_{ij}, \alpha, \beta\right).$$

Let $\delta=(\alpha, \beta)$.

Show that the log-conditional likelihood or conditional score function has the form

\[S^c(\delta) = \sum_{i=1}^{n}\sum_{j = (q+1)}^{m_i}\frac{\partial \mu_{ij}^c}{\partial \delta}{v_{ij}^c}^{-1}(y_{ij}-\mu_{ij}^c).\]

\end{Problem}

\pagebreak

\begin{Problem}[]{
Ordered Categorical data}


$Y$: ordinal response with categories labeled $1,2, \cdots, k$.


Let $$F(a \mid x)=P(Y \leq a \mid x),$$ where $a=1, \cdots,(k-1), x=\left(x_1, \cdots, x_p\right)^T$.

Proportional odds model: $$\logit F(a \mid x)=\theta_a+x^T \beta,\quad a=1, \cdots,(k-1).$$

Define $Y^*=\left(Y_1^*, \cdots, Y_{k-1}^*\right)$ with $Y_a^*=1_{(Y\leq a)}$. 

Then, $$\logit F(a \mid x)=\logit P\left(Y_a^*=1 \mid x\right).$$



\begin{center}
\begin{tabular}{c|cccccc}
    
    $Y$ & 1 & 2 & 3 & $\cdots$ & $k-1$ & $k$ \\
    \hline$Y_1^*$ & 1 & 0 & 0 & $\cdots$ & 0 & 0 \\
    $Y_2^*$ & 1 & 1 & 0 & $\cdots$ & 0 & 0 \\
    $\vdots$ & $\vdots$ & $\vdots$ & & & & $\vdots$ \\
    $Y_{k-1}^*$ & 1 & 1 & 1 & $\cdots$ & 1 & 0
  \end{tabular}
\end{center}


\dotfill


Example:


Assume that $$\logit P\left(Y_j \leq b \mid Y_{i \, j-1}=a\right)=\theta_\alpha+x_i{ }^T \beta_\alpha, \quad a, b=1, \cdots,(k-1).$$ 

It can be derived that

\[\logit P(Y_{ij}\leq b\mid Y_{i\,j-1}^* = y_{i\,j-1}^*) = \theta_b +\sum_{l=1 }^{k-1}\alpha_{ab}y_{i(j-1)l}^*+x_{ij}^T(\beta +\sum_{l=1}^{k-1}r_{l}y_{i(j-1)l}^*),\]

where
$\left\{\begin{array}{l}
\theta_{kb} = \theta_b,\\
\alpha_{lb} = \theta_{lb}-\theta_{k+1\,b},\\
\beta_k = \beta,\\
r_{l} = \beta_{l}-\beta_{l+1}
\end{array}\right..$


\end{Problem}


\pagebreak

\begin{Problem}[]{Log-linear transition models for count data}


 
$Y_{ij} \mid\left(H_{ij}, x_{ij}\right) \sim$ Poissom $\left(\mu^c_{ij}\right)$.

\dotfill

Model 1. Wong (1986) proposed that $$\mu_{ij}^c=\exp \left(x_{ij}^T\beta\right) \{1+\exp \left(-\alpha_0-\alpha_1 y_{i\,j-1}\right)\},$$ 

$\alpha_0, \alpha_1>0$, where $\beta$ is the influence of $x_{ij}$ as $y_{i\,j-1}=0$.

---

Remark. When $y_{i\,j-1}>0, \mu_{ij}^c$ decreases as $y_{i\,j-1}$ increases. A negative association is implied between the prior and current responses.

\dotfill

Model 2. $\mu_{ij}^c=\exp \left(x_{ij}^T\beta+\alpha y_{i\,j-1}\right)$.

---

Properties: 
\begin{enumerate}
  \item $\mu_{ij}^c$ increases as an exponential function of time as $\alpha>0$.
  \item When $\exp \left(x_{ij}^T \beta\right)=\mu$ and $\alpha<0$, it leads to a stationary process.
\end{enumerate}


\dotfill


Moded 3. $$\mu_{ij}=\exp \left(x_{ij}{ }^T \beta+\alpha\left\{\ln \left(y_{i\,j-1}^*\right)-x_{i\,j-1}^T \beta\right\}\right),$$ 

where $y_{i\,j-1}^*=\max \left\{y_{i\,j-1}, d\right\}, 0<d<1$.

---

Property: $\left\{\begin{array}{l}\alpha=0 \text { : it reduces to an oedinary log-tinear model. } \\ \alpha<0 \text { : negative correlation between } y_{i\,j-1} \text { and } y_{ij} \\ 
  \alpha>0 \text { : positive correlation between } y_{i\,j-1} \text { and } y_{ij}\end{array}\right.$


\end{Problem}


\begin{equation*}
  \begin{aligned}
  \end{aligned}
\end{equation*}


\end{document}

