\documentclass[UTF8,a4paper,10pt]{article}

% \documentclass[UTF8,a4paper,14pt]{article}
% \usepackage[utf8]{inputenc}
\usepackage{amsmath}
% \usepackage{algorithm,algorithmic}
\usepackage[linesnumbered,ruled,vlined]{algorithm2e}

% \usepackage{algorithmicx}
% \usepackage{algpseudocode}
\usepackage{hyperref}

% \usepackage{algpseudocode}
\usepackage{amssymb}
\usepackage{amsfonts}
%for 字體
%https://tug.org/FontCatalogue/
% \usepackage[T1]{fontenc}
% \usepackage{tgbonum}
% \usepackage[bitstream-charter]{mathdesign}
% \usepackage[T1]{fontenc}
% \usepackage{bm}#粗體
%\usepackage{boondox-calo}
\usepackage{textcomp}
\usepackage{fancyhdr}%导入fancyhdrf包
\usepackage{ctex}%导入ctex包
\usepackage{enumitem} %for在Latex使用條列式清單
\usepackage{varwidth}
\usepackage{soul} %for \ul
\usepackage{comment}%\begin{comment}\end{comment}
\usepackage{cancel}%\cancel{}
%\usepackage{unicode-math}

\usepackage[dvipsnames, svgnames, x_11names]{xcolor}

\usepackage[low-sup]{subdepth}
\usepackage{subdepth}

\newcommand{\indep}{\Perp \!\!\! \Perp}

\usepackage{amsthm}
\DeclareMathOperator{\E}{\mathbb{E}}
\DeclareMathOperator{\Var}{\textbf{Var}}
\DeclareMathOperator{\Cov}{\textbf{Cov}}
\DeclareMathOperator{\Cor}{\textbf{Cor}}
\DeclareMathOperator{\X}{\mathbf{X}}
\DeclareMathOperator{\Pro}{\mathbf{P}}
\DeclareMathOperator{\M}{\mathbf{M}}
\DeclareMathOperator{\Id}{\mathbf{I}}
\DeclareMathOperator{\Y}{\mathbf{Y}}
\DeclareMathOperator{\MSFE}{\mathbf{MSFE}}
\DeclareMathOperator{\e}{\mathbb{e}}
\DeclareMathOperator{\V}{\mathbf{V}} 
\DeclareMathOperator{\tr}{\text{tr}}
\DeclareMathOperator{\A}{\textbf{A}}
\DeclareMathOperator{\diag}{diag}
\DeclareRobustCommand{\rchi}{{\mathpalette\irchi\relax}}
\newcommand{\irchi}[2]{\raisebox{\depth}{$#1\chi$}} % inner command, used by \rchi
\DeclareMathOperator*{\argmax}{arg\,max}
\DeclareMathOperator*{\argmin}{arg\,min}

\DeclareMathSizes{20}{10}{10}{5}

\usepackage[a4paper, margin=1in]{geometry}
% \setlength\parskip{5ex}% it would be better define distance in ex (5ex) 
                         %  or in pt, pc, mm, etc (see edit below)

\setlength{\parindent}{0pt}
\usepackage{array, makecell} %


%中英文設定
%\usepackage{fontspec}
% \setmainfont{TeX Gyre Termes}
% \usepackage{xeCJK} %引用中文字的指令集
% %\setCJKmainfont{PMingLiU}
% \setCJKmainfont{DFKai-SB}





% \setmainfont{Times New Roman}
% \setCJKmonofont{DFKai-SB}
\pagenumbering{arabic}%设置页码格式
\pagestyle{fancy}
\fancyhead{} % 初始化页眉
\usepackage{advdate}

% \newcommand{\yesterday}{{\AdvanceDate[-1]\today}}

\fancyhead[C]{Longitudinal Data Analysis\quad HW 01\quad  R10A21126\quad  WANG YIFAN\quad   \today}
%\fancyhead[LE]{\textsl{\rightmark}}
%\fancyfoot{} % 初始化页脚
%\fancyfoot[LO]{奇数页左页脚}
%\fancyfoot[LE]{偶数页左页脚}
%\fancyfoot[RO]{奇数页右页脚}
%\fancyfoot[RE]{偶数页右页脚}

% \title{{Econometrics HW 05}}
% \author{R10A21126}
% \date{\today}

%\fancyhf{}
\usepackage{lastpage}
\cfoot{Page \thepage \hspace{1pt} of\, \pageref{LastPage}}

\renewcommand{\headrulewidth}{0.1pt}%分隔线宽度4磅
%\renewcommand{\footrulewidth}{4pt}

\allowdisplaybreaks
\usepackage[english]{babel}
%\usepackage{amsthm}
\newtheorem{theorem}{Theorem}[section]
\newtheorem{corollary}{Corollary}[theorem]
\newtheorem{lemma}[theorem]{Lemma}


\usepackage[most]{tcolorbox}

\definecolor{babyblue}{rgb}{0.54, 0.81, 0.94}

\newtcolorbox[auto counter]{mybox}[1]{
  % Define a new tcolorbox style with custom paragraph spacing
  before upper={\parskip=10pt},
    after upper={\parskip=10pt},
    enhanced,
    arc= 1 mm,boxrule=1.5pt,
    colframe=babyblue!80!pink,
    colback=white,
    coltitle=black,
    % colback=blue!5!white,
    attach boxed title to top left=
    {xshift=1.5em,yshift=-\tcboxedtitleheight/2},
    boxed title style={size=small,
    % frame hidden,
    colback=White},
    top=0.15in,
    % fonttitle=\bfseries,
    title= {#1},
    breakable
  }

\newtcolorbox[auto counter]{Problem}[2][]{
    enhanced,drop shadow={Pink!50!white},
    colframe=pink!80!white,
    fonttitle=\bfseries,
    title=Problem ~\thetcbcounter. #2,
    %separator sign={.},
    coltitle=black,
    colback=pink!15,
    top=0.15in,
    breakable
  }

\newenvironment{solution}
  {\renewcommand\qedsymbol{$\blacksquare$}\begin{proof}[Solution]}
  {\end{proof}}

\theoremstyle{definition}
\newtheorem{definition}{Definition}[section]

%\theoremstyle{notation}
\newtheorem*{notation}{\underline{Notation}}
%\newtheorem*{convention}{\underline{Convention}}
\newtheorem*{convention}{\underline{Convention}}

\theoremstyle{remark}
\newtheorem*{remark}{Remark}

\newenvironment{amatrix}[2]{%% [2] for 2 parameters 
  \left[\begin{array}
    %{cc\,|\,cc}
    %  {@{}*{#2}{c}\,|\,c*{#1}{c}}
     {{}*{#1}{c}\,|\,c*{#2}{c}}
}{%
  \end{array}\right]
}
% For augmented matrix  
%https://tex.stackexchange.com/questions/2233/whats-the-best-way-make-an-augmented-coefficient-matrix


% defines the paragraph spacing
\setlength{\parskip}{0.5em}


\usepackage[sorting=none, citestyle=verbose-inote,backref=true,ibidtracker=context,mincrossrefs=99,backend=biber, 
url = false,
doi = false, isbn=false,]{biblatex}

\addbibresource{R10A21126.bib}

\usepackage{graphicx}
\graphicspath{ {images/} }
\usepackage{caption}

% global change
\SetKwInput{KwData}{Input}
\SetKwInput{KwResult}{Output}
% https://tex.stackexchange.com/questions/299771/how-do-i-rename-data-from-kwdata-and-result-from-kwresult-in-begi

\hypersetup{hidelinks}

% \begin{equation*}
%   \begin{aligned}
%   \end{aligned}
% \end{equation*}

% \begin{mybox}{}
% \end{mybox}


% \begin{Problem}[]{}
% \end{Problem}

% \begin{solution}\,
% \end{solution}
  

% \begin{enumerate}[label=(\alph*)]
% \end{enumerate} 

% \setcounter{section}{3} 
% \setcounter{theorem}{3}

% \begin{theorem}\label{thm:3.4}
%   If $E = \bigcup_{k} E_k$ is a countable union of sets, then $|E|_e \leq \sum_{k} |E_k|_e$.
%   \end{theorem}

%  \footcite[][42]{Wheeden_Zygmund_2015}

\begin{document}


% \begin{mybox}{}


% \end{mybox}

\begin{Problem}[]{Testing for completely random dropouts}

    Let \(P_{ij}\) denote the probability that the \(i-\)th unit drops out at time \(t_j\), \(j = 1,\ldots,m\).

    Under the assumption of completely random dropouts, the probability \(P_{ij}\) may depend on time, treatment, or other explanatory variables, but cannot depend on the observed measurements \(y_{i} = (y_{i1}, \ldots, y_{i\,m_i})\).

    \section*{Testing Method:}
    \begin{enumerate}[label=(\alph*)]
        \item Choose the score function \(h_{k}(y_{1},\ldots,y_{k})\) so that extreme values constitute evidence against completely random dropouts. A sensible choice is 
        \[h_{k}(y_{1},\ldots,y_{k}) = \sum_{j=1}^{k}\omega_{j}y_{j}.\]
        \item For each of \(k = 1, \ldots, (m-1)\), define
        \begin{align*}
            R_{k} = \{i:m_i\geq k\},\\
            r_{k} = \{i:m_i = k\},
        \end{align*}
        and compute the set of scores \(h_{ik} = h_k(y_{i1\ldots,y_{ik}})\) for \(i\in R_{k}\).
        \item If \(1\leq |r_k|\leq |R_k|\), test the hypothesis that the \(r_k\)'s scores so defined are a random sample from the "populations" of \(R_k\)'s scores.
    \end{enumerate}
---

Remark:

\begin{enumerate}
    \item The implicit assumption that the separated \(p-\)values are mutually independent is valid precisely because once a unit drops out, it never returns. 
    \item A natural test statistics is \(\widebar{h}_k = \frac{1}{|r_k|} \sum_{\{j\in r_k\}} h_{jk}\). Under the assumption of completely random dropouts, 
    \[\widebar{h}_k \sim N\left(\widebar{H}_k, \frac{|R_k|-|r_k|}{(|R_k|-1)|r_k|}\sum_{\{j\in R_k\}} (h_{jk}-\widebar{H}_{k})^2/|R_k|\right),\]
    where \[\widebar{H}_{k} = \frac{1}{|R_k|} \sum_{\{j\in r_k\}} h_{jk}.\]
    \begin{itemize}
        \item When \(|R_k|\) or \(|r_k|\) is small, evaluate the randomization distribution of \(\widebar{h}_k\) under the null hypothesis.
        \item Alternative method ...
    \end{itemize}
    \item The Final stage consists of analyzing the resulting set of \(p-\)values via 
    \begin{enumerate}
        \item Empirical distribution of the \(p-\)values
        \item Kolmogorov-Smirnov statistic \(D_{+} = \sup |\hat{F}_n(p)-p|\)
    \end{enumerate}
\end{enumerate}
\end{Problem}

Given a finite population of size \(N\), with individual values \(\{X_i\}_{i=1}^{N}\), 

and a set of sample of size \(n\), drawn from the population without replacement, with values \(\{X_i\}_{i=1}^{n}\).

Let $\sigma^2$ be the population variance:
\[\sigma^2 = \Var[X_i] = \frac{1}{N}\sum_{i=1}^{N}(X_i-\mu),\]
where \(\mu = \sum_{i=1}^{N} X_i\) is the population mean.

Let \(\bar{X} = \frac{1}{n}S_n = \frac{1}{n}\sum_{i=1}^{n}X_i\) be the sample mean based on the sample set.



Since every pair $(X_i, X_j)$ for $i \neq j$ has the same joint distribution, we have
% the variance of the sum $S_n := X_1 + \ldots + X_n$ is
\begin{align*}
    \Var[S_n] = \sum_{i=1}^{n}\sum_{j=1}^{n}\Cov[X_i,X_j],
\end{align*}
where 
\[\Cov[X_i,X_j] = \begin{cases}
     \sigma^2 & i=j\\
    c  & i\neq j\\
\end{cases}.\]
Thus,
\begin{align*}
    \Var[S_n] 
    % &= n \Var[X_i]+ \left(n^2 - n\right)\Cov[X_i, X_j] \\
    &= n\sigma^2 + n(n-1)c.\label{eq.01}
\end{align*}
% \[
% \Var[S_n] = n \Var[X_i]+ \left(n^2 - n\right)\Cov(X_i, X_j) = n\sigma^2 + n(n-1)c.
% \]
% where we write $c$ for the covariance between the results of two distinct draws. Formula (1) 
which applies to the case $n=N$ as well. Notice that $S_N$ is a constant (equal to the sum of all $N$ values in the population). It follows that
\[
0 = \Var[S_N] = N\sigma^2 + N(N-1)c.
\]
Solve the equation above for $$c = -\frac{\sigma^2}{N-1}.$$

Hence,
\[
\Var[S_n] = n\sigma^2\left(1 - \frac{n-1}{N-1}\right) = \frac{N-n}{N-1} \cdot n\sigma^2
\]
and
\[
\Var[\bar{X}] = \frac{N-n}{N-1} \cdot \frac{\sigma^2}{n}.
\]

The factor \(\dfrac{N-n}{N-1}\) is the Finite Population Correction Factor (FPC).
% Notice the difference between formulas (4) and (5) and the corresponding formulas for sampling with replacement is a factor $\frac{N-n}{N-1}$, which is the famous correction factor for sampling without replacement.


\end{document}

