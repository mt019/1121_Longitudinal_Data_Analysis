\documentclass[UTF8,a4paper,10pt]{article}

% \documentclass[UTF8,a4paper,14pt]{article}
% \usepackage[utf8]{inputenc}
\usepackage{amsmath}
% \usepackage{algorithm,algorithmic}
\usepackage[linesnumbered,ruled,vlined]{algorithm2e}

% \usepackage{algorithmicx}
% \usepackage{algpseudocode}
\usepackage{hyperref}

% \usepackage{algpseudocode}
\usepackage{amssymb}
\usepackage{amsfonts}
%for 字體
%https://tug.org/FontCatalogue/
% \usepackage[T1]{fontenc}
% \usepackage{tgbonum}
% \usepackage[bitstream-charter]{mathdesign}
% \usepackage[T1]{fontenc}
% \usepackage{bm}#粗體
%\usepackage{boondox-calo}
\usepackage{textcomp}
\usepackage{fancyhdr}%导入fancyhdrf包
\usepackage{ctex}%导入ctex包
\usepackage{enumitem} %for在Latex使用條列式清單
\usepackage{varwidth}
\usepackage{soul} %for \ul
\usepackage{comment}%\begin{comment}\end{comment}
\usepackage{cancel}%\cancel{}
%\usepackage{unicode-math}

\usepackage[dvipsnames, svgnames, x_11names]{xcolor}

\usepackage[low-sup]{subdepth}
\usepackage{subdepth}

\newcommand{\indep}{\Perp \!\!\! \Perp}

\usepackage{amsthm}
\DeclareMathOperator{\E}{\mathbb{E}}
\DeclareMathOperator{\Var}{\textbf{Var}}
\DeclareMathOperator{\Cov}{\textbf{Cov}}
\DeclareMathOperator{\Cor}{\textbf{Cor}}
\DeclareMathOperator{\X}{\mathbf{X}}
\DeclareMathOperator{\Pro}{\mathbf{P}}
\DeclareMathOperator{\M}{\mathbf{M}}
\DeclareMathOperator{\Id}{\mathbf{I}}
\DeclareMathOperator{\Y}{\mathbf{Y}}
\DeclareMathOperator{\MSFE}{\mathbf{MSFE}}
\DeclareMathOperator{\e}{\mathbb{e}}
\DeclareMathOperator{\V}{\mathbf{V}} 
\DeclareMathOperator{\tr}{\text{tr}}
\DeclareMathOperator{\A}{\textbf{A}}
\DeclareMathOperator{\diag}{diag}
\DeclareRobustCommand{\rchi}{{\mathpalette\irchi\relax}}
\newcommand{\irchi}[2]{\raisebox{\depth}{$#1\chi$}} % inner command, used by \rchi
\DeclareMathOperator*{\argmax}{arg\,max}
\DeclareMathOperator*{\argmin}{arg\,min}

\DeclareMathSizes{20}{10}{10}{5}

\usepackage[a4paper, margin=1in]{geometry}
% \setlength\parskip{5ex}% it would be better define distance in ex (5ex) 
                         %  or in pt, pc, mm, etc (see edit below)

\setlength{\parindent}{0pt}
\usepackage{array, makecell} %


%中英文設定
%\usepackage{fontspec}
% \setmainfont{TeX Gyre Termes}
% \usepackage{xeCJK} %引用中文字的指令集
% %\setCJKmainfont{PMingLiU}
% \setCJKmainfont{DFKai-SB}





% \setmainfont{Times New Roman}
% \setCJKmonofont{DFKai-SB}
\pagenumbering{arabic}%设置页码格式
\pagestyle{fancy}
\fancyhead{} % 初始化页眉
\usepackage{advdate}

% \newcommand{\yesterday}{{\AdvanceDate[-1]\today}}

\fancyhead[C]{Longitudinal Data Analysis\quad HW 01\quad  R10A21126\quad  WANG YIFAN\quad   \today}
%\fancyhead[LE]{\textsl{\rightmark}}
%\fancyfoot{} % 初始化页脚
%\fancyfoot[LO]{奇数页左页脚}
%\fancyfoot[LE]{偶数页左页脚}
%\fancyfoot[RO]{奇数页右页脚}
%\fancyfoot[RE]{偶数页右页脚}

% \title{{Econometrics HW 05}}
% \author{R10A21126}
% \date{\today}

%\fancyhf{}
\usepackage{lastpage}
\cfoot{Page \thepage \hspace{1pt} of\, \pageref{LastPage}}

\renewcommand{\headrulewidth}{0.1pt}%分隔线宽度4磅
%\renewcommand{\footrulewidth}{4pt}

\allowdisplaybreaks
\usepackage[english]{babel}
%\usepackage{amsthm}
\newtheorem{theorem}{Theorem}[section]
\newtheorem{corollary}{Corollary}[theorem]
\newtheorem{lemma}[theorem]{Lemma}


\usepackage[most]{tcolorbox}

\definecolor{babyblue}{rgb}{0.54, 0.81, 0.94}

\newtcolorbox[auto counter]{mybox}[1]{
  % Define a new tcolorbox style with custom paragraph spacing
  before upper={\parskip=10pt},
    after upper={\parskip=10pt},
    enhanced,
    arc= 1 mm,boxrule=1.5pt,
    colframe=babyblue!80!pink,
    colback=white,
    coltitle=black,
    % colback=blue!5!white,
    attach boxed title to top left=
    {xshift=1.5em,yshift=-\tcboxedtitleheight/2},
    boxed title style={size=small,
    % frame hidden,
    colback=White},
    top=0.15in,
    % fonttitle=\bfseries,
    title= {#1},
    breakable
  }

\newtcolorbox[auto counter]{Problem}[2][]{
    enhanced,drop shadow={Pink!50!white},
    colframe=pink!80!white,
    fonttitle=\bfseries,
    title=Problem ~\thetcbcounter. #2,
    %separator sign={.},
    coltitle=black,
    colback=pink!15,
    top=0.15in,
    breakable
  }

\newenvironment{solution}
  {\renewcommand\qedsymbol{$\blacksquare$}\begin{proof}[Solution]}
  {\end{proof}}

\theoremstyle{definition}
\newtheorem{definition}{Definition}[section]

%\theoremstyle{notation}
\newtheorem*{notation}{\underline{Notation}}
%\newtheorem*{convention}{\underline{Convention}}
\newtheorem*{convention}{\underline{Convention}}

\theoremstyle{remark}
\newtheorem*{remark}{Remark}

\newenvironment{amatrix}[2]{%% [2] for 2 parameters 
  \left[\begin{array}
    %{cc\,|\,cc}
    %  {@{}*{#2}{c}\,|\,c*{#1}{c}}
     {{}*{#1}{c}\,|\,c*{#2}{c}}
}{%
  \end{array}\right]
}
% For augmented matrix  
%https://tex.stackexchange.com/questions/2233/whats-the-best-way-make-an-augmented-coefficient-matrix


% defines the paragraph spacing
\setlength{\parskip}{0.5em}


\usepackage[sorting=none, citestyle=verbose-inote,backref=true,ibidtracker=context,mincrossrefs=99,backend=biber, 
url = false,
doi = false, isbn=false,]{biblatex}

\addbibresource{R10A21126.bib}

\usepackage{graphicx}
\graphicspath{ {images/} }
\usepackage{caption}

% global change
\SetKwInput{KwData}{Input}
\SetKwInput{KwResult}{Output}
% https://tex.stackexchange.com/questions/299771/how-do-i-rename-data-from-kwdata-and-result-from-kwresult-in-begi

\hypersetup{hidelinks}

% \begin{equation*}
%   \begin{aligned}
%   \end{aligned}
% \end{equation*}

% \begin{mybox}{}
% \end{mybox}


% \begin{Problem}[]{}
% \end{Problem}

% \begin{solution}\,
% \end{solution}
  

% \begin{enumerate}[label=(\alph*)]
% \end{enumerate} 

% \setcounter{section}{3} 
% \setcounter{theorem}{3}

% \begin{theorem}\label{thm:3.4}
%   If $E = \bigcup_{k} E_k$ is a countable union of sets, then $|E|_e \leq \sum_{k} |E_k|_e$.
%   \end{theorem}

%  \footcite[][42]{Wheeden_Zygmund_2015}

\begin{document}


\begin{mybox}{Exponential Family}
  % \section{}
  \begin{itemize}
      \item Suppose $Y_1, \ldots, Y_n$ are independent random variables.
      \item Let $f(y_i; \theta_i, \phi)$ be the Probability Mass Function (PMF) or Probability Density Function (PDF) of $Y_i$, where $\phi$ is a scale parameter.
      \item If we can write
      \[
      f(y_i; \theta_i, \phi) = \exp\left(y_i \theta_i - b(\theta_i) \frac{1}{a(\phi)} + c(y_i, \phi)\right),
      \]
      then we call the PMF or the PDF $f(y_i; \theta_i, \phi)$ an exponential family.
  \end{itemize}
  


\end{mybox}


% Q1


  \begin{Problem}[]{}
    Find the form of GLM for the following distributions, and show the resonable link function:
    \begin{enumerate}
      \item Normal distributions
      \item Inverse Gaussian
      \item Binomial distribution
      \item Poisson distribution
      \item Gamma distribution
      \item Beta 
      % (重新參數化)
    \end{enumerate}
    
  \end{Problem}


  \begin{mybox}{Normal Distribution}
  % \section{Normal Distribution}
Assume $Y_i \sim \mathcal{N}(\mu_i, \sigma^2)$. Then, $E(Y_i) = \mu_i$ and $\sigma$ is a scale parameter. The Probability Density Function (PDF) is given by
\[
\frac{1}{\sqrt{2\pi\sigma}}e^{-\frac{(y_i - \mu_i)^2}{2\sigma^2}} = \exp\left\{\frac{y_i\mu_i - \mu_i^2/2}{\sigma^2} - \left(\frac{1}{2}\log(2\pi\sigma^2) - \frac{y_i^2}{2\sigma^2}\right)\right\}.
\]

Thus, use 

\begin{itemize}
  \item $\theta_i = \mu_i$,
  \item $b(\theta_i) = \frac{\theta_i^2}{2}$,
  \item $\phi = \sigma^2$, 
  \item $a(\phi) = \phi$,
  \item $c(y_i, \phi) = -\frac{1}{2}\log(2\pi\phi) - \frac{y_i^2}{2\phi}$.
\end{itemize}


\end{mybox}
\pagebreak


\begin{mybox}{Inverse Gaussian Distribution}
  Let us rewrite the probability density function (pdf) of the Inverse Gaussian distribution with parameters $\mu_i$ and $\lambda$:
  \[
  f(y_i; \mu_i, \lambda) = \left(\frac{\lambda}{2\pi y_i^3}\right)^{1/2}\exp\left\{-\frac{\lambda(y_i - \mu_i)^2}{2\mu_i^2 y_i}\right\}, \quad y > 0
  \]
  
  in the following form:
  \[
  \begin{aligned}
  f(y_i; \mu_i, \lambda) &= \exp\left\{-\frac{\lambda(y_i - \mu_i)^2}{2\mu_i^2 y_i} + \frac{1}{2}\ln\left(\frac{\lambda}{2\pi y_i^3}\right)\right\} \\
  &=\exp\left\{\frac{-\frac{1}{2\mu_i^2}y_i+\frac{1}{\mu_i}}{\frac{1}{\lambda}}+\left(\frac{1}{2}\ln \frac{\lambda}{2\pi y_i^3}-\frac{\lambda}{2y_i}\right) \right\}
  \end{aligned}
  \]
  
  Now, let's identify the exponential family components:
  \begin{itemize}
      \item Canonical parameter: $\theta_i = -\frac{1}{2\mu_i^2}$
      \item $b(\theta_i) = -\frac{1}{\mu_i} = -(-2\theta_i)^{\frac{1}{2}} $
      \item $\phi = \lambda$
      \item \(a(\phi) = \frac{1}{\phi}\)
      \item \(c(y_i,\phi) = \frac{1}{2}\ln\frac{\phi}{2\pi y_i^{3}}-\frac{\phi}{2y_i}\)
  \end{itemize}
  
  Thus, the Inverse Gaussian distribution can be shown to be a member of the exponential family.

\end{mybox}
  
  
\begin{mybox}{Binomial Distribution}
  % \section{}
Assume $Y_i \sim \text{Bin}(n_i, p_i)$. Then, $E(Y_i) = n_i p_i$. The Probability Mass Function (PMF) is given by
\[
\binom{n_i}{y_i} p_i^{y_i} (1 - p_i)^{n_i - y_i} = \exp\left\{y_i \log \left(\frac{p_i}{1 - p_i}\right) + n_i \log(1 - p_i) - \log \binom{n_i}{y_i}\right\}.
\]

Thus, 
\begin{itemize}
  \item $\theta_i = \log\left(\frac{p_i}{1 - p_i}\right)$, 
  \item $b(\theta_i) = n_i \log(1 + e^{\theta_i})$, 
  \item $\phi = 1$, $a(\phi) = 1$,
  \item $c(y, \phi) = -\log \binom{n_i}{y_i}$.
\end{itemize}


\end{mybox}

\pagebreak
  
\begin{mybox}{Poisson Distribution}
  % \section{}
Assume $Y_i \sim \text{Poisson}(\lambda_i)$. Then, $E(Y_i) = \lambda_i$. The Probability Mass Function (PMF) is given by
\[
\frac{\lambda_i^{y_i}e^{-\lambda_i}}{y_i!} = \exp\left\{y_i\log(\lambda_i) - \lambda_i - \log(y_i!)\right\}.
\]

Thus, 
\begin{itemize}
  \item $\theta_i = \log(\lambda_i)$, 
  \item $b(\theta_i) = e^{\theta_i}$, 
  \item $\phi = 1$, $a(\phi) = 1$, 
  \item $c(y_i, \phi) = -\log(y_i!)$.
\end{itemize}


\end{mybox}

  
\begin{mybox}{Gamma Distribution}
  % \section{}
Assume $x_i \sim \Gamma(\alpha, \beta_i)$, where $\beta_i$ is unknown. Then, $E(x_i) = \frac{\alpha}{\beta_i}$. The Probability Mass Function (PMF) is given by
\[
\frac{\beta_i^\alpha x_i^{\alpha-1} e^{-\beta_i x_i}}{\Gamma(\alpha)} = \exp\left\{\alpha\log x_i + \alpha\log(\beta_i) - \log(\Gamma(\alpha)) - \log(x_i) - \beta_i x_i\right\}.
\]

Assuming $\alpha$ is known, if we choose $y_i = x_i$, then 
\begin{itemize}
  \item $\theta_i = -\beta_i$ ($\theta_i < 0$), 
  \item $b(\theta_i) = -\alpha\log(-\theta_i)$, 
  \item $\phi = 1$, and $a(\phi) = 1$.
\end{itemize}

Remark: We can also choose 
$y_i = -x_i$ and $\theta_i = \beta_i$. In this case, $b(\theta_i) = -\alpha\log(\theta_i)$.

\end{mybox}

% \begin{mybox}{Beta Distribution}
% % \section{}
% Assume $x_i \sim \text{Beta}(\alpha, \beta_i)$, where $\beta_i$ is unknown. Then, the expected value of $x_i$ is $E(x_i) = \frac{\alpha}{\alpha + \beta_i}$. The Probability Mass Function (PMF) is given by

% \begin{equation*}
%   \begin{aligned}
% &\frac{\Gamma(\alpha + \beta_i)}{\Gamma(\alpha)\Gamma(\beta_i)} x_i^{\alpha - 1}(1 - x_i)^{\beta_i - 1}\\ 
% = &\exp\left\{\log(\Gamma(\alpha + \beta_i)) - \log(\Gamma(\alpha)) - \log(\Gamma(\beta_i)) + (\alpha - 1)\log(x_i) + (\beta_i - 1)\log(1 - x_i)\right\}.
% \end{aligned}
% \end{equation*}

% Assuming $\alpha$ is known, if we choose $y_i = x_i$, then $\theta_i = \beta_i$, $b(\theta_i) = -\log(\theta_i)$, $\phi = 1$, and $a(\phi) = 1$.

% Remark: We can also choose $y_i = 1 - x_i$ and $\theta_i = \alpha$. In this case, $b(\theta_i) = -\log(\theta_i)$.
% \end{mybox}


\pagebreak

  \begin{Problem}[]{}
  Paper Summarization
    
  \end{Problem}

  The part 1 of this page talks about:

- **Sliced inverse regression (SIR)**: A novel data-analytic tool for reducing the dimension of the input variable x without fitting any parametric or nonparametric model¹[1]. It explores the inverse view of regression, where x is regressed against y, and uses a simple step function to estimate the inverse regression curve²[2].
- **Effective dimension reduction (e.d.r.) space**: The linear space generated by the unknown row vectors {3k (k = 1, ..., K) in the model y = f({3lx, ..., {3Kx, e), where f is an arbitrary unknown function. The goal is to estimate this space, which captures all the information about y from x.
- **Inverse regression curve**: The curve E(x I y) that connects the conditional mean of x given y as y varies. Under certain conditions, this curve falls into the e.d.r. space. A principal component analysis on the covariance matrix of the estimated inverse regression curve can locate its main orientation, yielding the estimates for e.d.r. directions³[3].
- **Sampling properties of SIR**: The output of SIR provides root n consistent estimates for the e.d.r. directions under a design condition on the distribution of x⁴[4]. The eigenvalues of the covariance matrix can be used to assess the number of components in the model and the effectiveness of SIR.
- **Simulation results**: SIR is demonstrated to be effective in reducing the dimension of x from 10 to 2 for a data set with 400 observations. The spin-plot of y against the projected variables obtained by SIR mimics the spin-plot of y against the true directions very well⁵[5]. A chi-squared statistic is proposed to test whether a direction found by SIR is spurious⁶[6].


\pagebreak
\begin{Problem}[]{}
  Describe
  \begin{enumerate}
    \item how to conduct the EM (Expectation-Maximization) algorithm
    \item how to conduct  MCMC
  \end{enumerate}
    
  \end{Problem}
  Denoting
  \[ Q(q|q_0, x) = \mathbb{E}_{q_0} \left[ \log L_c(q|x,Z) \right], \]
  the EM algorithm indeed proceeds "iteratively" by maximizing \(Q(q|q_0, x)\) at each iteration, and, if \(q^{\hat{(1)}}\) is the value of \(q\) maximizing \(Q(q|q_0, x)\), by replacing \(q_0\) by the updated value \(q^{\hat{(1)}}\). In this manner, a sequence of estimators \(\{q^{\hat{(j)}}\}_j\) is obtained, where
  \[ Q(q^{\hat{(j)}}|q^{\hat{(j-1)}}) \]

  Pick a starting value \(q^{\hat{(0)}}\) and set \(m = 0\).
Repeat
\begin{enumerate}
  \item Compute (the E-step) \[ Q(q|q^{\hat{(m)}, x}) = \mathbb{E}_{q^{\hat{(m)}}} \left[ \log L_c(q|x,Z) \right], \]where the expectation is with respect to \(k(z|q^{\hat{(m)}, x})\).
  \item Maximize \(Q(q|q^{\hat{(m)}, x})\) in \(q\) and take (the M-step)   \[ q^{\hat{(m+1)}} = \arg\max_q Q(q|q^{\hat{(m)}, x}) \] and set \(m = m+1\).
\end{enumerate}

until a fixed point is reached; i.e., \(q^{\hat{(m+1)}} = q^{\hat{(m)}}\).

  

\dotfill

Gibbs Sampling of MCMC (Markov Chain Monte Carlo)



\begin{Problem}[]{}
  Under the regularity conditions,
  \begin{equation*}
    \begin{aligned}
      n^{-\frac{1}{2}}\big[\begin{pmatrix} \hat{\beta}\\ \hat{\alpha}\end{pmatrix} - \begin{pmatrix} \beta\\ \alpha\end{pmatrix}  \big]\overset{d}{\to } N(0,\Sigma),
    \end{aligned}
  \end{equation*} 
  where \(Sigma\) can be estimated By
  \begin{equation*}
    \begin{aligned}
    \end{aligned}
  \end{equation*} 
    
  \end{Problem}
  

\end{document}